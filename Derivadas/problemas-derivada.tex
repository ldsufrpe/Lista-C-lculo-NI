\begin{enumerate}
    \item Use a definição para encontrar a derivada em relação a $x$ da função dada.
        \begin{enumerate}
        \begin{multicols}{2}
            \item $f(x)=mx+b$
            \item $F(t)=4t-t^2$
            \item $f(x)=\sqrt{16-x}$
            \item $f(x)=\dfrac{x+3}{2x-1}$
            \end{multicols}
        \end{enumerate}

    \item Determine se a afirmação é falsa (F) ou verdadeira (V). Se for verdadeiro, justifique. Caso contrário, explique o motivo ou dê uma exemplo que mostre que é falsa.
    \begin{enumerate}
        \item[(\quad )] Se $f$ é derivável em $x=0$ então $f'$ também será.
        \item[(\quad)] Se $f$ é contínua em $x=a$ então $f$ é derivável em $a$.
        \item[(\quad )] Se $h(x)=\left(x^6-x^4\right)^5$, então $h^{(31)}(x)=0$.
        \item[(\quad)] \textcolor{red}{mais...}
    \end{enumerate}
    
    

    \item Use as regras de derivação e calcule a derivada das funções abaixo.
        \begin{enumerate}
            \begin{multicols}{2}
            \item $f(s)=3.1415$
            \item $f(x)=x^5+2x^3+x^2+1$
            \item $f(x)=(x-1)(x^2+3x)$
            \item $f(x)=\dfrac{2x^2+2}{2x-1}$
            \item $f(t)=\dfrac{\sqrt{2t}-t}{t^3}$
            \item $f(x)=x^2\sqrt{20-x^4}$
            \item $f(x)=\dfrac{x^2+3x+1}{\sqrt{x}}$
            \item $p(r)= e^r + r^e$
            \item $r(x)=e^{x^2}$
            \item $l(x)=5^{x^2-2x}$
            \item $y=\sqrt{x}e^x$
            \item $h(s)=\left(s+\dfrac{1}{s}\right)^3$
            \item $y= \left(\dfrac{x-1}{x+1}\right)^3$
            \item $y=\dfrac{\sqrt{t^2 +1}}{\sqrt{t^2+4}}$
            \end{multicols}
        \end{enumerate}
        
    \item Encontre as derivadas de até à terceira ordem da função $f(x)=x^4-2x^3+ \frac{1}{2}x^2 + x +1$.
    \item Encontre as derivadas de até à sétima ordem da função $f(x)=e^x$.
    \item Seja 
    \begin{align}
        g(x)=\begin{cases}
                x^2+ 2 & \text{ se } x<1,\\
                -x + 4 &\text{ se } x \geq 1.\\
            \end{cases}
    \end{align}
        \begin{enumerate}
            \item $g'(1)$ existe? Ou seja, $g$ é derivável em 1.
            \item $g$ é contínua em 1?
            \item Esboce o gráfico de $g$ e $g'$.
        \end{enumerate}
\item Seja        
    \begin{align}
        g(x)=\begin{cases}
                2 &\text{ se } x\geq 0\\
                x^2 + 2 &\text{ se } x < 0\\
            \end{cases}
    \end{align}
        \begin{enumerate}
            \item $g'(0)$ existe? Ou seja, $g$ é derivável em 0.
            \item $g$ é continua em 0? 
            \item Esboce o gráfico de $g$ e $g'$.
        \end{enumerate}
    \item Considere 
         \begin{align}
        h(x)=\begin{cases}
                x^2 &\text{ se } x\geq 1,\\
                mx+b &\text{ se } x \geq 1.\\
            \end{cases}
    \end{align}
    Encontre o valor de $m$ e $b$ que tornem  $h$ derivável em $I=(-\infty, +\infty)$.
    \item Considere $f(x)=|x^2+16|$.
        \begin{enumerate}
            \item $f$ é derivável em $I=(-\infty, +\infty)$?
            \item Esboce os gráficos de $f$ e $f'$.
        \end{enumerate}

     \item Determine $\lim\limits_{x\to 1}\dfrac{x^{10^4}-1}{x-1}$.
    
     \item Calcule a derivada de $g(x)=\left(1+ \frac{1}{x}\right)^x$.
     \item Calcule a derivada de $h(x)=x^{\sin{x}}$.
     \item Calcule a derivada de $f(x)=\left(\cos{x}\right)^x$.
    \item Determina a derivadas das funções a seguir.
        \begin{enumerate}
        \begin{multicols}{2}
            \item $f(x)=\sin{3x}$
            \item $f(x)=\sec{(0.5x)}$
            \item $f(x)=\sin{(3x)}\cos{(x^2)}$
            \item $f(x)=\cos{(x^2-1)}$
            \item $f(x)=2\sin{x}+\sin^2{2x}$
            \item $y=\dfrac{1}{\cot{x}}$
            \item $y=2^{\tan{(\sqrt{x})}}$
            \item $y=e^{\tan{(\sqrt{x})}}$
            \item $y=e^x\cos{(x^2-1)}$
            \item $y=\tan(\sin{3x})$
            \item $y=(2+\sin{x})^{\cos{2x}}$
            \item $y=\dfrac{x^2\sin^{-1}{x}}{\cos{3x}}$
            \end{multicols}
        \end{enumerate}
    
    \item Encontre $f'(x)$ sabendo-se que 
            \begin{align*}
                \dfrac{d}{dx}\left[f(3x)\right]=x^2.
            \end{align*}
    
    \item A notação $\dfrac{d^{n}y}{dx^n}$ indica a n-ésima derivada (ou derivada de ordem $n$) em relação a variável $x$ da função $y=f(x)$. 
    \begin{enumerate}
        \item Se $y=xe^{-x}$ encontre uma fórmula para $\dfrac{d^{n}y}{dx^n}$.
        \item Calcule $\dfrac{d^{n}y}{dx^n}$ para $n=1000$.
    \end{enumerate}
    
     \item Uma equação é chamada de \textbf{Equação diferencial} quando envolve uma função desconhecida $y$ e suas derivadas $y', y'', y'''$,  etc. Encontre um função desconhecida $y$ que satisfaça:
         \begin{enumerate}
            \begin{multicols}{2}
             \item $y'=y$
             \item $y'=\cos{x}$
             \item $y''=-y$
             \item $y'=x^3$
            \end{multicols}
            \end{enumerate}
    \item Mostre que a função $y=\dfrac{1+e^x}{1-e^x}$ satisfaz a equação diferencial $y'=\frac{1}{2}(y^2-1)$.
   
    \item Encontre a derivada das funções.
        \begin{enumerate}
        \begin{multicols}{2}
            \item $y=x^2\sin^{-1}{x}$
            \item $y=\sin^{-1}{\left(\cos^{-1}{\left(x^2\right)}\right)}$
            \item $y=\dfrac{x^2\sin^{-1}{x}}{\cos{3x}}$
            \item $y=\tan^{-1}{(x^2+1)}$
            \end{multicols}
        \end{enumerate}
        
    \item Derive a função.
        \begin{enumerate}
            \begin{multicols}{2}
            \item $ k(x)=(2-x^2)\ln{x}$
            \item $q(x)=1-\dfrac{\ln{x}}{x}$
            \item $h(x)=x^2\ln{x} + 2e^x$
            \item $y=\ln\left[x^2\sqrt{1+x}+1\right]$
            \item $y=\ln\left[\dfrac{x^2-1}{x-1}\right]$
            \item $y=\log_2\left(e^{-x}\cos{2x}\right)$
            \end{multicols}
        \end{enumerate}
        
    \item Encontre $\dfrac{dy}{dx}$ por derivação implícita.
        \begin{enumerate}
            \begin{multicols}{2}        
            \item $x^4 + y^4=1$
            \item $y^3+\sin{xy}=2$
            \item $y^x + x = y^2$
            \item $e^y + xy=x$
            \item $x\cos{y}+y\cos{x}=1$
            \item $\tan{(x-y)}=\dfrac{y}{1+x^2}$
            \end{multicols}
        \end{enumerate}
    \item Encontre equações para reta tangente e para reta normal à curva no ponto dado.
        \begin{enumerate}
            \item $x^2+2xy + y^2=1$, \quad $(1,-2)$
            \item $y=(10 +x)e^{-x}$, \quad $(0,10)$
        \end{enumerate}
        
    \item Determine a reta que é tangente ao gráfico de $f(x)=x^3+3x$ e  paralela à reta $y=6x-1$.
    
    \item Determine uma reta que seja paralela a $x+y=1$ e que seja tangente à curva $x^2+xy+y^2=3$.
    \item Em qual ponto sobre a curva $y=\left[\ln(x+4)\right]^2$ a reta tangente é horizontal.
    
    \item Encontre uma equação da reta tangente à curva $y=e^x$ que passe pelo ponto $(0,0)$.
    \item Encontre os pontos sobre a elipse $x^2+2y^2=1$ onde a reta tangente tem inclinação 1.
\end{enumerate}
\begin{center}
\textbf{Aplicações à Química}
\end{center}

As questões que seguem, foram retiradas do livro ``Matemática Ensino Médio Vol 1" de Luiz Roberto Dante.
\begin{enumerate}
\item A radioatividade é um fenômeno que ocorre em núcleos de átomos instáveis por emitirem partículas e radiações. A medida de tempo na qual metade da quantidade do material radioativo se desintegra é denominada meia-vida ou período de semidesintegração ($P$). A cada período de tempo $P$ a quantidade de material radioativo cai à metade da anterior, sendo possível relacionar a quantidade de material radioativo a qualquer tempo com a quantidade inicial por meio de uma função exponencial: 
$$N(t)=N_0 . \left(\dfrac{1}{2}\right)^{\frac{t}{p}},$$
em que $N_0$ é a quantidade inicial do material radioativo, $t$ é o tempo decorrido e $P$ é o valor da meia-vida do material radioativo considerado. A radioatividade faz parte de nossa vida, como quando se faz uma tomografia. Um dos isótopos mais usados nos radiofármacos injetados nos pacientes submetidos à tomografia é o carbono 11, cuja meia-vida é de 20 minutos. Qual o tempo necessário, em minutos, para que uma amostra de carbono 11 se reduza a $\dfrac{1}{4}$ do que era quando foi obtida?

\item O carbono 14 é um isótopo raro do carbono presente em todos os seres vivos. Com a morte, o nível de C14 no corpo começa a decair. Como é um isótopo radioativo de meia-vida de 5730 anos, e como é relativamente fácil saber o nível original de C14 no corpo dos seres vivos, a medição da atividade de C14 em um fóssil é uma técnica muito utilizada para datações arqueológicas. A atividade radioativa do C14 decai com o tempo pós-morte segundo a função exponencial
\[A(t)=A_0 . \left(\dfrac{1}{2}\right)^{\frac{t}{5730}},\]
em que $A_0$ é a atividade natural do C14 no organismo vivo e $t$ é o tempo decorrido em anos após a morte. Suponha que um fóssil encontrado em uma caverna foi levado ao laboratório para ter sua idade estimada. Verificou-se que emitia 7 radiações de C14 por grama/hora. Sabendo que o animal vivo emite 896 radiações por grama por hora, qual é a idade aproximada desse fóssil?

\item Os átomos de um elemento químico radioativo têm uma tendência natural a se desintegrar (emitindo partículas e se transformando em outros elementos). Dessa forma, com o passar do tempo, a quantidade original desse elemento diminui. Chamamos de meia-vida o tempo que o elemento radioativo leva para desintegrar metade de sua massa radioativa. O antibiótico acetilcefuroxina apresenta meia-vida de 3 horas. Se uma pessoa tomou 50mg desse medicamento, qual é a quantidade de antibiótico ainda presente no organismo:
\begin{enumerate}
\item após 12 horas de sua ingestão?
\item após t horas de sua ingestão?
\end{enumerate}

\item Considere uma substância radioativa de meia-vida $P$ que inicia o processo de desintegração. Que porcentagem de sua massa ainda restará após metade de sua primeira meia-vida?

\item Quando se administra um remédio, sua concentração no organismo deve oscilar entre dois níveis, pois não pode ser tão baixa a ponto de não fazer efeito (\emph{Ce}) e não pode ser tão alta a ponto de apresentar efeitos indesejáveis (toxicidade) ao paciente (\emph{Cp}). Quando, após um certo tempo depois de ministrado o remédio, o nível de concentração no organismo atinge \emph{Ce}, toma-se mais uma dose do remédio a fim de elevar o nível de concentração para \emph{Cp}. Esse tempo entre as administrações das doses é chamado de \emph{tempo interdoses}. É importante notar que o tempo interdoses, após a primeira medição, é o tempo que decorre para a concentração máxima tolerada \emph{Cp} decair até a concentração mínima eficaz \emph{Ce}.

Lembrando que a concentração de uma droga no organismo, após um tempo \emph{t}, é dada por
\[C(t)=C_0 . \left(\dfrac{1}{2}\right)^{\frac{t}{p}},\]
em que $C_0$ é a quantidade inicial ingerida do remédio, \emph{t} é o tempo decorrido e \emph{P} é o valor da meia-vida da substância no organismo, obtenha em função de \emph{Ce, Cp e P}: (Use que $\log 2=0,30$)
\begin{enumerate}
\item o valor do tempo interdoses;
\item a concentração de remédio $D$ nas doses que devem ser administradas ao paciente a cada intervalo interdoses.
\end{enumerate}
\end{enumerate}

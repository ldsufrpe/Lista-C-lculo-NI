\begin{enumerate}


\item Considere as funções definidas por $f(x)=\sqrt{\dfrac{2x-4}{-x^2+3x}}$ \quad \text{e}\quad $g(x) = 3-\sqrt{x+1}$.

\begin{enumerate}
\item Indique os domínios das funções $f$ e $g$.
\item Calcule os zeros de $f$, $g$ e determine, caso existam, $f(0)$ e $g(0)$.
\item Indique a imagem de $g$.
\item Indique os domínios de: $f+g$ e $\frac{f}{g}$.
\end{enumerate}

\item Considere as funções definidas por $f(x)=\sqrt{x}$, $g(x)=x^2$, $h(x)=\frac{1}{x}$.
\begin{enumerate}
    \item Indique seus domínios.
    \item Determine suas imagens.
    \item Indique, justificando, se as funções são injetivas e/ou sobrejetivas.
\end{enumerate}

\item Considere as funções definidas por $f(x)=\sqrt{x}$, $g(x)=x^2$, $h(x)=\dfrac{1}{x-1}$. Indique os domínios e as expressões analíticas de $g\circ f$, $f\circ g$, $h\circ f$, $f\circ h$.

\item Suponha $g$ uma função par e seja $h = f \circ g.$
  \begin{enumerate}
    \item Dê um exemplo para mostrar que $h$ nem sempre é uma função ímpar.
    \item Mostre que se $f$ é ímpar então $h$ também será.
    \item  O que podemos afirmar sobre a paridade de  $h$ se $f$  for uma função par? Justifique.
    \end{enumerate}


\item Considere as funções definidas por $f(x)=\ln{x+2}$ e $g(x)=\ln(x-2)-2$.
\begin{enumerate}
    \item Indique seus domínios.
    \item Qual é a intersecção com o eixo $x$ do gráfico de $f$ e $g$?
\end{enumerate}

\item Considere a função  definida por $f(x)=\ln{(e^x-3)}$.
\begin{enumerate}
    \item Indique o domínio de $f$.
    \item Determine $f^{-1}$ e indique seu domínio.
\end{enumerate}

\item Se $f(x)=2x + \ln{x}$ , encontre $f^{-1}(2)$.

\item Estude a monoticidade das seguintes funções:
\begin{enumerate}
    \item $f(x)=-x^3+1$
    \item $f(x)=\dfrac{1}{|x|+2}$
\end{enumerate}
\item Considere a função $f:\mathbb{R}\setminus\{1\}\longrightarrow \mathbb{R}$, definida por $f(x)=\dfrac{e^x}{x-1}$.
\begin{enumerate}
    \item Estude a monoticidade de $f$.
    \item $f$ é limitada? Ou seja, existe $M>0$ tal que $-M\leq f(x) \leq M$ para todo $x\neq 1$? 
\end{enumerate}


\item Considere as funções definidas por 
$$f(x)=\left\{\begin{array}{cl}
         2x   & \mbox{ se } x\leq -1    \\
         -x+1 & \mbox{ se } x>-1
       \end{array}
\right.
\quad \text{  e  }\quad
g(x) = \begin{cases}
\sqrt{x-5} & \text{ se } x>5\\
6-2x & \text{ se } x<5
\end{cases}
.$$

\begin{enumerate}
    \item Determine os domínios das funções $f$ e $g$.
    \item Indique a imagem de $g$.
    \item Esboce os gráficos de $f$ e $g$.
\end{enumerate}


\item Indique \textbf{Verdadeiro} ou \textbf{Falso} nas afirmações a seguir. Justifique todas as respostas.
\begin{enumerate}
\item[( )] Toda função estritamente crescente é injetora.
\item[( )] A função $f(x)=2^x$ é a inversa de $h(x)=\log_3x$.
\item[( )] A imagem da função $\ln (x+1)$ é negativa no intervalo $(0,1)$.
\item [( )]A função injetiva $f:A\longrightarrow f(A)$ é bijetiva.
\end{enumerate}


\item Indique quais das seguintes funções são pares ou ímpares
ou nem par nem impar:
\begin{enumerate}
\begin{multicols}{2}
\item $f(x)= 2 + |5x|$
\item $f(x) = \dfrac{x^3 - x}{x^2 + 1}$ 
\item $f(x) =\dfrac{x - 1}{x + 1}$
\item $f(x)=x|x|$
\item $f(x)=\sqrt{x}$
\item $f(x)=\begin{cases}
x^2+1 &\text{ se }  x<0\\
2x & \text{ se }  x>0
\end{cases}$
\end{multicols}
\end{enumerate}

\item Considere a função $f(x)=|x-1|+|x-2|.$
\begin{enumerate}
\item Indique o domínio de $f$.
\item Mostre que $f(x)=\left\{\begin{array}{lcl}
         -2x+3   & \mbox{ se } & x\leq 1    \\
         1 		 & \mbox{ se } & 1<x<2\\
         2x-3 	 & \mbox{ se } & x\geq 2
       \end{array}
\right..$
\item Esboce o gráfico de $f$.
\end{enumerate}



\item Esboce o gráfico de:
\begin{enumerate}
\begin{multicols}{2}
    \item $y = 2 + |5x-1|$
    \item $y= |x^2-x-1|$
    \item $y=1+\dfrac{1}{|x+1|}$
    \item $y=\dfrac{1}{|x| +2}$
    \item $y=||x|-1|$ 
    \item $y=\dfrac{|x|}{x}$
\end{multicols}
\end{enumerate}

\item Esboce o gráfico de:
\begin{enumerate}
\begin{multicols}{2}
    \item $y=-2^x$
    \item $y=2-\frac{1}{3}e^{-x}$
    \item $y=e^{|x|}$
    \item $y=2(1-e^x)$
\end{multicols}
\end{enumerate}

\item Esboce o gráfico de :
\begin{enumerate}
\begin{multicols}{2}
     \item $y=\ln{(-x)}$
    \item $y=-\ln{x}$
    \item $y=\ln{x+2}$
    \item $y=\ln(x-2)-2$
    
\end{multicols}
\end{enumerate}




\item Determine a equação da reta que: passa pelo ponto dado e que seja paralela à reta dada:
\begin{enumerate}
    \item  passa pelo ponto $(1,3)$ e é paralela à reta $y=2x+3$
    \item passa pelo ponto$(-1,2)$ e é paralela à reta $x-y$
\end{enumerate}
\item Determine a equação da reta que:  dado e que seja perpendicular à reta dada:
\begin{enumerate}
    \item passa pelo ponto $(1,2)$ e é perpendicular à reta  $y=x$.
    \item passa pelo ponto $(0,1)$ e é perpendicular à reta $5x+y$
\end{enumerate}

\item Esboce o gráfico de cada função a seguir:
\begin{enumerate}
\begin{multicols}{2}
    \item $y=4\sin{2x}$
    \item $y=1-\sin({x +\pi/2})$
    \item $y=\frac{1}{2}\left(1-\cos{x}\right) $
    \item $y=1+|\cos{x}|$
\end{multicols}
\end{enumerate}

\item Sabendo que
\begin{align*}
\sin(\frac{\pi}{2}+a)=\frac{5}{13} &\qquad \frac{3\pi}{2}\leq a \leq 2\pi,\\
\tan(7\pi -b)=\frac{4}{3} & \qquad \frac{\pi}{2}\leq b\leq \pi,
\end{align*}
calcule:
\begin{enumerate}
\begin{multicols}{3}
    \item $\sin(a+b)$
    \item $\cos{(a-b)}$
    \item $\cos(\frac{\pi}{4}+b)$
    \end{multicols}
\end{enumerate}

\item Use a identidade 
$$\sin^2{x}+\cos^2{x}=1$$
para mostrar que $1+\tan^2{x}=\sec^2{x}$.

\item Use as identidades

\begin{align*}
    \sin(x+ y)=\sin{x}\cos{y}+ \cos{x}\sin{y},\\
\cos(x+y) = \cos{x}\cos{y}- \sin{x}\sin{y},
\end{align*}


para mostrar que 
\begin{enumerate}
\begin{multicols}{3}
    \item $\cos{2x} =2\cos^2{x}-1 $
    \item  $\sin{2x}=2\sin{x}\cos{x}$
    \item $\sin{(x+\frac{\pi}{2})}=\cos{x}$
    \item $\cos{(x+\frac{\pi}{2})}=-\sin{x}$
    \item $\sin^2x=\frac{1}{2}\left(1-\cos{2x}\right)$
    \item $\cos^2x=\frac{1}{2}\left(1+\cos{2x}\right)$
    \end{multicols}
\end{enumerate}


\end{enumerate}


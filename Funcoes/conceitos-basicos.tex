\subsection*{Conceitos Básicos}
\begin{tcolorbox}

\begin{defi} Uma função de variável real a valores reais é uma função $f:X\longrightarrow \mathbb{R}$, onde $X\subset \mathbb{R}$. O conjunto $X$ é o domínio de $f$ e indica-se por $D_f$, ou seja, $D_f=X$. O conjunto dos números reais é o contradomínio. A imagem é o subconjunto  de $\mathbb{R}$, indicado por $\textrm{Im}_f$ e definido por $f(X)=\textrm{Im}_f=\{y\in\mathbb{R}: y=f(x), x\in X\}$.
\end{defi}

\begin{defi}Seja $f:X\longrightarrow \mathbb{R}$ uma função. O conjunto $G_f=\left\{(x,f(x))\in\mathbb{R}^2: x\in X\right\}$ denomina-se \textbf{gráfico} de $f$.
\end{defi}



\begin{defi}

Uma função $f: X \rightarrow \mathbb{R}$  é dita \textbf{injetora} (injetiva)  se para cada $y\in\mathrm{Im}_f,$ existe um único $x\in D_f=X$ tal que $f(x)=y.$
Ou seja, se $$f(x_1)=f(x_2)\Rightarrow x_1=x_2, \forall x_1, x_2\in\mathrm{D}_f.$$
\end{defi}
% \begin{defi}Função composta
% \begin{figure}
%     \centering
%     \begin{tikzpicture}[ scale=0.5]
\clip(-4.3,-3.2) rectangle (10.14,6.3);
\draw (-3,2)-- (-1,2);
\draw (0,2)-- (2,2);
\draw (3,2)-- (5,2);
\draw [shift={(-0.46,1.6)}] plot[domain=0.73:2.53,variable=\t]({1*1.23*cos(\t r)+0*1.23*sin(\t r)},{0*1.23*cos(\t r)+1*1.23*sin(\t r)});
\draw [shift={(2.48,1.66)}] plot[domain=0.56:2.5,variable=\t]({1*1.2*cos(\t r)+0*1.2*sin(\t r)},{0*1.2*cos(\t r)+1*1.2*sin(\t r)});
\draw [shift={(0.94,3.84)}] plot[domain=3.79:5.66,variable=\t]({1*3.72*cos(\t r)+0*3.72*sin(\t r)},{0*3.72*cos(\t r)+1*3.72*sin(\t r)});
\draw (-2.92,1.9) node[anchor=north west] {A};
\draw (0.1,1.86) node[anchor=north west] {B};
\draw (3.08,1.84) node[anchor=north west] {C};
\draw (2.54,3.34) node[anchor=north west] {f};
\draw (-0.58,3.28) node[anchor=north west] {g};
\draw (0.7,0.32) node[anchor=north west] {$f\circ g$};
\begin{scriptsize}
\fill [color=black] (-3,2) circle (0.5pt);
\fill [color=black] (-1,2) circle (0.5pt);
\fill [color=black] (0,2) circle (0.5pt);
\fill [color=black] (2,2) circle (0.5pt);
\fill [color=black] (3,2) circle (0.5pt);
\fill [color=black] (5,2) circle (0.5pt);
\fill [color=black,shift={(0.46,2.42)},rotate=180] (0,0) ++(0 pt,2.25pt) -- ++(1.95pt,-3.375pt)--++(-3.9pt,0 pt) -- ++(1.95pt,3.375pt);
\fill [color=black,shift={(3.5,2.3)},rotate=180] (0,0) ++(0 pt,2.25pt) -- ++(1.95pt,-3.375pt)--++(-3.9pt,0 pt) -- ++(1.95pt,3.375pt);
\fill [color=black,shift={(3.96,1.68)}] (0,0) ++(0 pt,2.25pt) -- ++(1.95pt,-3.375pt)--++(-3.9pt,0 pt) -- ++(1.95pt,3.375pt);
\end{scriptsize}
\end{tikzpicture}
% \end{figure}
%\end{defi}
\begin{defi}
Seja $f: X \longrightarrow \mathbb{R}$ uma injetora. Então, a sua inversa $f^{-1}: f(X) \longrightarrow X$ é definida por  
\begin{align*}
    f^{-1}(y) \iff f(x)=y \quad  \forall y\in f(X)
\end{align*}
\end{defi}

\begin{defi}
Uma função $f: X \rightarrow \mathbb{R}$ é dita \textbf{sobrejetora} (sobrejetiva) se $f(X)=\mathbb{R}$.
\end{defi}
\begin{defi}
Uma função $f: X \rightarrow \mathbb{R}$ é dita \textbf{bijetora} (bijetiva) se é injetora e sobrejetora.
\end{defi}



\begin{defi}
Uma função $f: X \longrightarrow\mathbb{R}, X\subset\mathbb{R},$ é
dita \textbf{monótona} se ela preserva (ou inverte) a relação de
ordem. Sejam $x_1,x_2\in \mathbb{R}$, dizemos que $f$ é monótona
\begin{enumerate}
\item [(i)] \textbf{crescente} ou \textbf{estritamente crescente}, quando
$x_1<x_2 \Rightarrow f(x_1)<f(x_2)$
\item [(ii)]\textbf{decrescente} ou \textbf{estritamente decrescente}, quando
$x_1<x_2 \Rightarrow f(x_1)>f(x_2)$
\item [(iii)] \textbf{não-decrescente}, quando
$x_1<x_2 \Rightarrow f(x_1)\leq f(x_2)$
\item [(iv)]\textbf{monótona não-crescente}, quando
$x_1<x_2 \Rightarrow f(x_1)\geq f(x_2)$
\end{enumerate}
\end{defi}
\end{tcolorbox}

\begin{tcolorbox}

\subsubsection*{Paridade de Funções}
\begin{defi} Se $D_{f} \subseteq \mathbb{R}$ simétrico em relação à origem. Então $f$ é dita \textbf{par} se  $f(-x) = f(x), \forall x \in D_f$ e \textbf{impar} se $f(-x) = -f(x), \forall x \in D_f$.

\begin{figure}[H]
\centering
\subfigure{
    \begin{tikzpicture}[scale=0.5]
\begin{axis}[
 axis lines=middle,
 ticklabel style={fill=white},
%  xmin=-4,xmax=4,
 ymin=-1,ymax=1,
 xlabel=$x$,ylabel=$y$,
 domain=-4:4,
 samples=100,
 smooth,
 xtick={-1.5,0,1.5}, ytick={-1,1},
 xticklabels={$-a$,0,$a$}, yticklabels={,,},
 width=10cm]
\coordinate  (x2) at (1,1);
\coordinate  (x1) at (-1,-1);
\addplot[blue] {0.5*sin(deg(x))};

\draw[dashed] (-1.5,0) --(-1.5,{0.5*sin(deg(-1.5)}) --(0,{0.5*sin(deg(-1.5)}) ;

\draw[dashed] (1.5,0)  -- (1.5,{0.5*sin(deg(1.5)})--(0, {0.5*sin(deg(1.5)});
\node[] at (axis cs:2.2,0.7) {função impar};

\node[right] at (axis cs:0, {0.5*sin(deg(-1.5))}) {$-f(a)$};
\node[left] at (axis cs:0, {0.5*sin(deg(1.5))}) {$f(a)$};
\end{axis}
\end{tikzpicture}}
    \qquad
\subfigure{
\begin{tikzpicture}[scale=0.5]
\begin{axis}[
 axis lines=middle,
 ticklabel style={fill=white},
%  xmin=-4,xmax=4,
 ymin=-0.5,ymax=2,
 xlabel=$x$,ylabel=$y$,
 domain=-4:4,
 samples=100,
 smooth,
 xtick={-1.5,0,1.5}, ytick={1},
 yticklabels={,,},   xticklabels={$-a$,0,$a$},
 width=10cm]
\coordinate  (x2) at (1,1);
\coordinate  (x1) at (-1,-1);
\addplot[blue] {1+0.5*cos(deg(x))};

\draw[dashed] (-1.5,0) --(-1.5,{1+0.5*cos(deg(-1.5)}) -- (1.5,{1+0.5*cos(deg(1.5)})--(1.5,0) ;
\node[] at (axis cs:2.2,1.5) {função par};
\end{axis}
\end{tikzpicture}}
\end{figure}
\end{defi}
\end{tcolorbox}
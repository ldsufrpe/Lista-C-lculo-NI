
Todas as suas afirmações devem ser justificadas.

\begin{enumerate}

\item Calcule os limites, usando as propriedades de limites:
\begin{multicols}{3}
\begin{enumerate}
\item $ \displaystyle\lim_{x\longrightarrow 0} (3 -7x -5x^2)$
\item $\displaystyle\lim_{x\longrightarrow 3} (3x^2 -7x + 2)$
\item $\displaystyle\lim_{x\longrightarrow -1} (-x^5 + 6x^4 + 2)$
\item $\displaystyle\lim_{x\longrightarrow \frac{1}{2}} (2x + 7)$
\item $ \displaystyle\lim_{x\longrightarrow -1} [(x + 4)^3(x +
2)^{-1}]$ \item $ \displaystyle\lim_{x\longrightarrow 0}
[(x-2)^{10}(x + 4)]$ \item $ \displaystyle\lim_{x\longrightarrow
2} \displaystyle\frac{x + 4}{ 3x- 1}$
 \item
$ \displaystyle\lim_{t\longrightarrow 2} \displaystyle\frac{t +
3}{t + 2} $ \item $ \displaystyle\lim_{x\longrightarrow 1}
\displaystyle\frac{x^2-1}{ x- 1}$ \item $
\displaystyle\lim_{t\longrightarrow 2} \displaystyle\frac{t^2 + 5t
+ 6}{ t + 2} $ \item $\displaystyle\lim_{t\longrightarrow 2}
\displaystyle\frac{t^2 -5t + 6}{ t -2}$ \item
$\displaystyle\lim_{s\longrightarrow \frac{1}{ 2}}
\displaystyle\frac{s + 4}{ 2s} $ \item
$\displaystyle\lim_{x\longrightarrow 4}\displaystyle\sqrt[3]{2x +
3}$ \item $\displaystyle\lim_{x\longrightarrow 7} (3x + 2)^{2/3}$
\item $\displaystyle\lim_{x\longrightarrow \sqrt{ 2}}
\displaystyle\frac{2x^2 - x} {3x}$ \item
$\displaystyle\lim_{x\longrightarrow 2}
\displaystyle\frac{x\sqrt{x} - \sqrt{2}}{ 3x - 4}$ \item
$\displaystyle\lim_{x\longrightarrow -\frac{1}{ 3}} (2x +
3)^{\frac{1}{4}}$
\end{enumerate}
\end{multicols}


\item Seja $f(x)$ a função definida pelo gráfico:
%\begin{center}
%\includegraphics[width=0.8\textwidth, clip = true, trim = 0 220 0 %180]{Questao7_Lista8.pdf}
%\end{center}
Encontre, se existir:
\begin{multicols}{3}
\begin{enumerate}
\item $\displaystyle\lim_{x\longrightarrow3-}f(x)$ \item
$\displaystyle\lim_{x\longrightarrow3+} f(x)$ \item
$\displaystyle\lim_{x\longrightarrow3} f(x) $ \item $
\displaystyle\lim_{x\longrightarrow-\infty} f(x)$ \item
$\displaystyle\lim_{x\longrightarrow+\infty} f(x)$ \item
$\displaystyle\lim_{x\longrightarrow4} f(x).$
\end{enumerate}
\end{multicols}
Escreva a expressão da função.


\item Seja $$f(x) = \left\{\begin{array}{rcl} x - 1,& \mbox{para}&
x \leq 3\\
 3x-7,& \mbox{para}& x
> 3 \end{array}
\right.$$ Calcule:
\begin{multicols}{3}
\begin{enumerate}
\item $\displaystyle\lim_{x\longrightarrow3-}f(x)$ \item
$\displaystyle\lim_{x\longrightarrow3+} f(x)$ \item
$\displaystyle\lim_{x\longrightarrow3} f(x) $ \item $
\displaystyle\lim_{x\longrightarrow5-} f(x)$ \item
$\displaystyle\lim_{x\longrightarrow5+} f(x)$ \item
$\displaystyle\lim_{x\longrightarrow5} f(x).$
\end{enumerate}
\end{multicols}

Esboce o gráfico de $f(x).$


\item Seja $f(x) = 2 + |5x-1|.$ Calcule os limites indicados se
existirem:
\begin{multicols}{3}
\begin{enumerate}
\item $\displaystyle\lim_{x\longrightarrow\frac{1}{ 5}-} f(x)$
\item $ \displaystyle\lim_{x\longrightarrow\frac{1}{ 5}+} f(x)$
\item $\displaystyle\lim_{x\longrightarrow\frac{1}{ 5}} f(x).$
\end{enumerate}
\end{multicols}
Esboce o gráfico de $f(x).$

\item Para cada uma das seguintes funções, ache $\displaystyle\lim_{x\longrightarrow2} \frac{f(x)-f(2)}{x-2}.$
\begin{multicols}{3}

\begin{enumerate}
\item $f(x)=3x^2$
\item $f(x)=\displaystyle\frac{1}{x},$ $x\ne0$
\item $f(x)=\displaystyle\frac{2}{3}x^2$
\item $f(x)=3x^2+5x-1$
\item $f(x)=\displaystyle\frac{1}{x+1},$ $x\ne-1$
\item $f(x)=x^3$
\end{enumerate}
\end{multicols}

\item Calcule os limites:
\begin{multicols}{2}
\begin{enumerate}
\item $\displaystyle\lim_{x\longrightarrow-1}
\displaystyle\frac{x^3 + 1}{ x^2 - 1}$ \item $
\displaystyle\lim_{t\longrightarrow-2} \displaystyle\frac{t^3 +
4t^2 + 4t}{ (t + 2)(t -3)}$ \item $
\displaystyle\lim_{x\longrightarrow2} \displaystyle\frac{x^2 + 3x
- 10}{ 3x^2 - 5x - 2}$ \item
$\displaystyle\lim_{t\longrightarrow5} \displaystyle\frac{2t^2 -
3t - 5} {2t - 5}$ \item $\displaystyle\lim_{x\longrightarrow a}
\displaystyle\frac{x^2 + (1 - a)x - a}{ x - a}$ \item $
\displaystyle\lim_{x\longrightarrow4} \displaystyle\frac{3x^2 -
17x + 20}{ 4x^2 - 25x + 36}$ \item $
\displaystyle\lim_{x\longrightarrow-1} \displaystyle\frac{x^2 + 6x
+ 5}{ x^2 - 3x - 4}$ \item $
\displaystyle\lim_{x\longrightarrow-1} \displaystyle\frac{x^2 -
1}{ x^2 + 3x + 2} $ \item $\displaystyle\lim_{x\longrightarrow2}
\displaystyle\frac{x^2 - 4}{ x - 2}$ \item $
\displaystyle\lim_{x\longrightarrow2} \displaystyle\frac{x^2 - 5x
+ 6}{ x^2 - 12x + 20}$ \item $\lim_{h\longrightarrow0}
\displaystyle\frac{(2 + h)^4 - 16}{ h}$ \item $
\displaystyle\lim_{t\longrightarrow0} \displaystyle\frac{(4 + t)^2
- 16}{ t}$ \item $\displaystyle\lim_{t\longrightarrow0}
\displaystyle\frac{\displaystyle\sqrt{25 + 3t} - 5}{ t}.$ \item
$\displaystyle\lim_{t\longrightarrow 0}\displaystyle\frac{
\displaystyle\sqrt{a^2 + bt} -a }{t},\quad a
> 0 $ \item $\displaystyle\lim_{h\longrightarrow 1}\displaystyle\frac{
\displaystyle\sqrt{h} - 1}{ h - 1}$ \item
$\displaystyle\lim_{h\longrightarrow-4} \displaystyle\frac{
\displaystyle\sqrt{ 2(h^2 - 8)} + h}{ h + 4} $ \item
$\displaystyle\lim_{h\longrightarrow0} \displaystyle\frac{
\displaystyle\sqrt[3]{ 8 + h}-2 }{h} $ \item
$\displaystyle\lim_{x\longrightarrow0}\displaystyle\frac{
\displaystyle\sqrt{1 + x}-1}{-x} $ \item
$\displaystyle\lim_{x\longrightarrow0}\displaystyle\frac{
\displaystyle\sqrt{x^2 + a^2} -a }{ \displaystyle\sqrt{x^2 + b^2}-
b}\quad a,b > 0$ \item $\displaystyle\lim_{x\longrightarrow
a}\displaystyle\frac{ \displaystyle\sqrt[3]{ x}-
\displaystyle\sqrt[3]{ a}}{ x - a},\quad a \ne 0$ \item
$\displaystyle\lim_{x\longrightarrow1}\displaystyle\frac{
\displaystyle\sqrt[3]{ x} -1}{\displaystyle\sqrt[4]{ x} -1} $
\item $\displaystyle\lim_{x\longrightarrow1} \displaystyle\frac{
\displaystyle\sqrt[3]{x^2} -2 \displaystyle\sqrt[3]{ x} + 1}{ (x -
1)^2}$ \item
$\displaystyle\lim_{x\longrightarrow 4}\displaystyle\frac{
 3-\displaystyle\sqrt{ 5 + x}}{ 1 -\displaystyle\sqrt{5 -x}} $ \item
$\displaystyle\lim_{x\longrightarrow0}\displaystyle\frac{\displaystyle\sqrt{1+
x}- \displaystyle\sqrt{ 1- x}}{ x}.$
\end{enumerate}
\end{multicols}
\end{enumerate}
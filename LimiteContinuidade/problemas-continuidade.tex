
\begin{enumerate}

\item Verifique se a função dada é contínua no(s) valor(es) de $a$ indicado(s).
\begin{enumerate}
\begin{multicols}{2}
\item $f(x)=3x^2-3x-2$ em $a=2$
\item $f(x)=x^5-x^2+x-5$ em $a=0$
\item $f(x)=\displaystyle\frac{x+2}{x+1}$ em $a=1$ e $a=0$
\item $f(x)=\displaystyle\frac{2x+1}{3x-6}$ em $a=2$ e $a=1$
\item $f(x)=\displaystyle\frac{\sqrt{x}-2}{x-4}$ em $x=2$ e $a=4$
\item $f(x)=\left\{\begin{array}{rcl}
    x+1&\mathrm{se}&x\leq 2\\
    2&\mathrm{se}&x> 2
    \end{array}\right.$ em $a=2.$

\item $f(x)=\left\{\begin{array}{rcl}
    x^2+1&\mathrm{se}&x\leq 3\\
    2x+4&\mathrm{se}&x> 3
    \end{array}\right.$ em $a=3.$

\item $f(x)=\left\{\begin{array}{rcl}\displaystyle\frac{x^2-1}{x+1}&\mathrm{se}&x<-1\\
    x^2-3&\mathrm{se}&x\geq -1
    \end{array}\right.$ em $a=-1$
    \end{multicols}
\end{enumerate}

\item Seja $$f(x) =
\left\{\begin{array}{rcl}\displaystyle\sqrt{-x},& \mbox{se}&x
< 0\\
 3-x,& \mbox{se}&0 \leq x <3\\
 (x-3)^2,& \mbox{se}& x > 3
  \end{array} \right.$$

\begin{enumerate} \item Calcule cada limite, se ele existir:
\begin{multicols}{3}
\begin{enumerate}
\item $\displaystyle\lim_{x\longrightarrow0+} f(x)$ \item
$\displaystyle\lim_{x\longrightarrow0-} f(x) $ \item $
\displaystyle\lim_{x\longrightarrow0} f(x)$ \item
$\displaystyle\lim_{x\longrightarrow3-} f(x)$ \item
$\displaystyle\lim_{x\longrightarrow3+} f(x) $ \item $
\displaystyle\lim_{x\longrightarrow3} f(x).$
\end{enumerate}
\end{multicols}
\item Onde $f$ é descontínua?
\item Esboce o gráfico de $f$.
\end{enumerate}

\item Seja $$g(x) =
\left\{\begin{array}{rcl}2x-x^2,& \mbox{se}&0\leq x\leq 2\\
 2-x,& \mbox{se}&2<x \leq 3\\
  x-4,& \mbox{se}& 3<x<4\\
\pi,& \mbox{se}& x \geq 4
  \end{array} \right.$$

\begin{enumerate} \item Para cada um dos números 2, 3 e 4 descubra se $g$ é contínua à esquerda, à direita e/ou contínua.
\item Esboce o gráfico de $g$.
\end{enumerate}
\item Determine, se existirem, os pontos onde a seguinte função não é contínua.
$$f(x)=\displaystyle\sqrt{(3-x)(6-x)}$$

\item Investigue a continuidade nos pontos indicados:
%\begin{multicols}{3}
\begin{enumerate}
\item $f(x)=x-|x|,$ em $x=0.$
 \item $ g(x) = \left\{\begin{array}{rcl}
 1 - x^2,& \mbox{se}& x < 1\\
  1-|x|,& \mbox{se}& x> 1\\
  1,& \mbox{se}& x = 1 \end{array}
\right.$ em $x = 1.$
 \item $h(x) = \left\{\begin{array}{rcl} x^2,& \mbox{se}&x \geq -1\\
  1 -|x|,& \mbox{se}& x < -1\end{array} \right.$ em $x = 1.$
\end{enumerate}
%\end{multicols}

\item Investigue a continuidade nos pontos indicados:
\begin{enumerate}
 \item $ f(x) =\left\{\begin{array}{rcl}
\dfrac{x^3- 8}{ x^2 -4},& \mbox{se}& x \ne2\\
 3,& \mbox{se}& x = 2
\end{array}
\right.$ em $x = 2.$

\item $ f(x) = \dfrac{x^2 - 3x + 7}{ x^2 + 1}$ em $x =
2.$ \item $ f(x) = \dfrac{2}{ 3x^2 + x^3 - x - 3}$ em
$x = -3.$
\end{enumerate}

\item Determine, se existirem, os valores de $x \in
D_f$, nos quais a função não é contínua:
\begin{enumerate}
\item $f(x) = \left\{\begin{array}{rcl} \dfrac{x}{ x^2
- 1},& \mbox{se}& x^2 \ne 1\\
 0 ,& \mbox{se}& x=-1
\end{array} \right.$
\item $f(x) = \dfrac{x -|x|}{ x}$ 
\item $f(x)
=  \left\{\begin{array}{rcl}\dfrac{ x^2 -3x +
4}{ x - 1} ,& \mbox{se}& x \ne 3\\
 1,& \mbox{se}& x = 1
 \end{array} \right.$
\end{enumerate}

\item Faça o gráfico e analise a continuidade das seguintes
funções:
\begin{enumerate}
 \item $ h(x) =
\left\{\begin{array}{rcl} \dfrac{x^2 - 4}{ x + 2} ,&
\mbox{se}&x \ne -2\\
1,& \mbox{se}& x = -2\end{array} \right.$ \item $h(x) =
\left\{\begin{array}{rcl} \dfrac{x}{|x|},&
\mbox{se}&x \ne 0\\
 -1,& \mbox{se}& x = 0\end{array}
\right.$ \item $f(x) =
\dfrac{x^3 + 3x^2 - x - 3}{ x^2 + 4x + 3}.$
\end{enumerate}



\item Determine, se existirem, os valores de $x \in
D_f,$ nos quais a função abaixo não é contínua:
$$ h(x) = \left\{\begin{array}{rcl}\displaystyle\sqrt{x^2 + 5x + 6},& \mbox{se}& x < -3 \quad \mbox{e} \quad x
> -2\\
-1,& \mbox{se}& -3\leq x \leq -2\end{array} \right. .$$

\item Faça o gráfico e analise a continuidade da seguinte
função:
$$h(x) = \left\{\begin{array}{rcl} 0,& \mbox{se}& x \leq
0\\
x,& \mbox{se}&x > 0\end{array} \right. .$$

\item Considere a função $f$, de  domínio  $[0,2\pi]$ definida por
$$
f(x) = 
\begin{cases}
 1+\ln{(\pi -x)}& \text{  se  } 0\leq x\leq \pi\\
 \cos{2x}& \text{  se  }  \pi \leq x\leq 2\pi
\end{cases}
$$
\begin{enumerate}
    \item Estude $f$ quanto à continuidade.
    \item Determine os zeros de $f$.
%    \item Seja $\theta\in [0,2\pi]$ tal que $\cos{\theta}=\frac{2}{3}$. Determine $f(\theta)$.
\end{enumerate}


\item Calcule $p$ de modo que as funções abaixo sejam contínuas:
\begin{enumerate}
\item $ h(x) = \left\{\begin{array}{rcl} x^2 + px + 2,&
\mbox{se}&x\ne 3\\
 3,& \mbox{se}&x = 3\end{array} \right.$
\item $h(x) = \left\{\begin{array}{rcl} x + 2p,& \mbox{se}&x
\leq -1\\
 p^2,& \mbox{se}& x
> -1\end{array} \right.$
\end{enumerate}


                                                                                                                                                                                      
\item Indique, usando os Teoremas \ref{operacoes}, \ref{todas}, \ref{composta}, o conjunto onde as funções abaixo são contínuas.
\begin{enumerate}
\begin{multicols}{3}
\item $f(x)=\dfrac{2x}{x^2+5x+6}$
  \item $f(x)=|-1+\cos 2x|$
\item $f(x)=3-\sqrt{x+1}$
\item $f(x)=\sqrt{2 + \dfrac{2}{x}}$
\item $f(x)=\dfrac{\sin{x}}{x+1}$
\item $f(x)=\dfrac{\tan x}{\sqrt{4-x^2}}$
\end{multicols}
\end{enumerate}

\item Mostre que $f(x)=\dfrac{2x^2+5}{x-2}$ é contínua em $(2,+\infty)$. 

\item Se $f(x) =x^2+10\sin{x}$, mostre que existe um número $c$ tal que $f(c)=1.000.$
\item Use o \textbf{Teorema do Valor Intermediário} para mostrar que existe uma raiz da equação dada no intervalo especificado.
\begin{enumerate}
\begin{multicols}{2}
    \item $x^4+x-3=0$, \quad $(1,2)$
    \item $2x^3+x^2+2=0$, \quad $(-2,-1)$
    \item $e^x=3-2x$, \quad $(0,1)$
    \item $x^2=\sqrt{x+1}$, \quad $(1,2)$
\end{multicols}
\end{enumerate}

\item Em cada um dos casos seguintes, dê um exemplo de uma função contínua num intervalo $[a, b]$ tal que $f(a).f(b)<0$ e:
   \begin{enumerate}

    \item Tenha uma única raiz em 1.

    \item $f$ tenha exatamente 2 raízes.

    \item $f$ tenha uma infinidade de raízes.
   \end{enumerate}

% final
\end{enumerate}
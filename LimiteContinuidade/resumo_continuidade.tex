\begin{tcolorbox}
\begin{defi}Uma função é \textbf{contínua} em um número a se $\lim\limits_{x \to a}f(x)=f(a)$.
\end{defi}
\begin{defi}Uma função é \textbf{contínua à direita} em  um número a se $\lim\limits_{x \to a^+}f(x)=f(a)$ e \textbf{contínua à esquerda} de a se $\lim\limits_{x \to a^-}f(x)=f(a)$ 
\end{defi}
\begin{defi}
Uma função $f$ é contínua em um intervalo se for contínua em todos
os números do intervalo. (Se $f$ for definida somente de um lado da extremidade do intervalo, entendemos continuidade na extremidade como continuidade à direita ou à esquerda.)
\end{defi}

\begin{teorema}\label{operacoes}Se $f$ e $g$ são contínuas em $a$ então as funções abaixo também são contínuas em a:
\begin{multicols}{3}
\begin{enumerate}
    \item $f+g$
    \item $f-g$
    \item $f\cdot g$
    \item $\dfrac{f}{g}$ se $g(a)\neq 0$
    \item $c\cdot f$
\end{enumerate}
\end{multicols}
\end{teorema}
\begin{teorema}\label{todas} Os tipos de funções a seguir são contínuas em todo número de seus domínios:
\begin{multicols}{2}
\begin{enumerate}
    \item Funções polinomiais. 
    \item Funções trigonométricas 
    \item Funções exponenciais.
    \item Funções logarítmicas.
\end{enumerate}
\end{multicols}
\end{teorema}
\begin{teorema}\label{composta}
Se $g$ for contínua em $a$ e $f$ em $g(a)$  então $f(g(x))$ é contínua em $a$.
\end{teorema}
\end{tcolorbox}

\begin{tcolorbox}
\begin{teorema}[Teorema do valor intermediário]Se $f$ for contínua em $[a,b]$ e se $d$ for um número real entre $f(a)$ e $f(b)$, então existe pelo menos um $c$ em $[a,b]$ tal que $f(c)=d$.
\end{teorema}
\begin{figure}[H]
    \centering
    \begin{tikzpicture}[scale=0.6]
\coordinate (a) at (0.5, 0);

\begin{axis}[
 axis lines=middle,
 ticklabel style={fill=white},
 xmin=-0.5,xmax=3,
 ymin=-0.25,ymax=2,
 xlabel=$x$,ylabel=$y$,
 domain=0.5:2.5,
 samples=100,
 smooth, 
 yticklabels={,,},
 xticklabels={,,},
 width=10cm
 ]
\coordinate (a) at (0.5, 0);
\coordinate (b) at (2.5, 0);
\coordinate (fa) at (0, {0.25+1.5*exp(-(0.5-1)^2)});
\coordinate (fb) at (0, {0.25+1.5*exp(-(2.5-1)^2)});
\coordinate (d) at (0, {0.25+1.5*exp(-(1.8-1)^2)});
\coordinate (c) at (1.8, 0);

\addplot[red, domain=0:2.5, line width=1pt] {0.25+1.5*exp(-(1.8-1)^2)};
\addplot[blue, line width=1pt] {0.25+1.5*exp(-(x-1)^2)};


\draw[dashed] (a) -- (0.5,{0.25+1.5*exp(-(0.5-1)^2)});
\draw[dashed] (fa) -- (0.5,{0.25+1.5*exp(-(0.5-1)^2)});
\draw[dashed] (b) -- (2.5,{0.25+1.5*exp(-(2.5-1)^2)});
\draw[dashed] (fb) -- (2.5,{0.25+1.5*exp(-(2.5-1)^2)});
\draw[dashed] (c) -- (1.8,{0.25+1.5*exp(-(1.8-1)^2)});

%\node[circle,draw=black, fill=black, inner sep=0pt,minimum size=5pt] at (0.5,{0.25+1.5*exp(-(0.5-1)^2)});
\fill[black] (0.5,{0.25+1.5*exp(-(0.5-1)^2)})  circle (1.5pt);
\fill[black] (1.8,{0.25+1.5*exp(-(1.8-1)^2)})  circle (1.5pt);
\fill[black] (2.5,{0.25+1.5*exp(-(2.5-1)^2)})  circle (1.5pt);


\node[below] at (a) {$a$};
\node[left] at (fa) {$f(a)$};
\node[below] at (b) {$b$};
\node[left] at (fb) {$f(b)$};
\node[left] at (d) {$d$};
\node[below] at (c) {$c$};
\end{axis}
\end{tikzpicture}
\end{figure}
\end{tcolorbox}